\documentclass{article}
\usepackage[utf8]{inputenc}
\author{Kevin L}

\usepackage{amsmath,amsthm,amssymb,amsfonts,graphicx,framed,indentfirst}
\usepackage[normalem]{ulem}
\usepackage[italicdiff]{physics}
\usepackage[T1]{fontenc}
%\usepackage{pifont} %For unusual symbols
%\usepackage{mathdots} %For unusual combinations of dots
\usepackage{wrapfig}
\usepackage{lmodern,mathrsfs}
\usepackage[inline,shortlabels]{enumitem}
\setlist{topsep=2pt,itemsep=2pt,parsep=0pt,partopsep=0pt}
\usepackage[dvipsnames]{xcolor}
\usepackage[utf8]{inputenc}
\usepackage[a4paper, top=0.5in,bottom=0.2in, left=0.5in, right=0.5in, footskip=0.3in, includefoot]{geometry}
\usepackage[most]{tcolorbox}
\usepackage{tikz,tikz-3dplot,tikz-cd,tkz-tab,tkz-euclide,pgf,pgfplots}
\pgfplotsset{compat=newest}
\usepackage{multicol}
\usepackage[bottom,multiple]{footmisc} %ensures footnotes are at the bottom of the page, and separates footnotes by a comma if they are adjacent
% \usepackage[backend=bibtex,style=numeric]{biblatex}
% \renewcommand*{\finalnamedelim}{\addcomma\addspace} % forces authors' names to be separated by comma, instead of "and"
% \addbibresource{bibliography}
\usepackage{hyperref}
\usepackage[nameinlink]{cleveref} %nameinlink ensures that the entire element is clickable in the pdf, not just the number
\usetikzlibrary{positioning}
\newcommand{\ep}{\varepsilon}
\newcommand{\N}{\mathbb{N}}
\newcommand{\Q}{\mathbb{Q}}
\newcommand{\R}{\mathbb{R}}
\newcommand{\C}{\mathbb{C}}
\newcommand{\Z}{\mathbb{Z}}
\renewcommand{\iff}{\Leftrightarrow}
\newcommand{\dom}{\text{dom }}
\DeclareMathOperator{\range}{\text{range}}
\usepackage{mathtools}
\DeclarePairedDelimiter\ceil{\lceil}{\rceil}
\DeclarePairedDelimiter\floor{\lfloor}{\rfloor}

\newtheoremstyle{mystyle}{}{}{}{}{\sffamily\bfseries}{.}{ }{}
\newtheoremstyle{cstyle}{}{}{}{}{\sffamily\bfseries}{.}{ }{\thmnote{#3}}
\makeatletter
\renewenvironment{proof}[1][\proofname] {\par\pushQED{\qed}{\normalfont\sffamily\bfseries\topsep6\p@\@plus6\p@\relax #1\@addpunct{.} }}{\popQED\endtrivlist\@endpefalse}
\makeatother

\theoremstyle{mystyle}{\newtheorem{definition}{Definition}[section]}
\theoremstyle{mystyle}{\newtheorem{proposition}[definition]{Proposition}}
\theoremstyle{mystyle}{\newtheorem{theorem}[definition]{Theorem}}
\theoremstyle{mystyle}{\newtheorem{lemma}[definition]{Lemma}}
\theoremstyle{mystyle}{\newtheorem{corollary}[definition]{Corollary}}
\theoremstyle{mystyle}{\newtheorem*{remark}{Remark}}
\theoremstyle{mystyle}{\newtheorem*{remarks}{Remarks}}
\theoremstyle{mystyle}{\newtheorem*{example}{Example}}
\theoremstyle{mystyle}{\newtheorem*{examples}{Examples}}
\theoremstyle{definition}{\newtheorem*{exercise}{Exercise}}
\theoremstyle{cstyle}{\newtheorem*{cthm}{}}
\theoremstyle{cstyle}{\newtheorem*{clemma}{}}

\tcolorboxenvironment{definition}{boxrule=0pt,boxsep=0pt,colback={red!10},left=8pt,right=8pt,enhanced jigsaw, borderline west={2pt}{0pt}{red},sharp corners,before skip=10pt,after skip=10pt,breakable}
\tcolorboxenvironment{proposition}{boxrule=0pt,boxsep=0pt,colback={Orange!10},left=8pt,right=8pt,enhanced jigsaw, borderline west={2pt}{0pt}{Orange},sharp corners,before skip=10pt,after skip=10pt,breakable}
\tcolorboxenvironment{theorem}{boxrule=0pt,boxsep=0pt,colback={blue!10},left=8pt,right=8pt,enhanced jigsaw, borderline west={2pt}{0pt}{blue},sharp corners,before skip=10pt,after skip=10pt,breakable}
\tcolorboxenvironment{lemma}{boxrule=0pt,boxsep=0pt,colback={Cyan!10},left=8pt,right=8pt,enhanced jigsaw, borderline west={2pt}{0pt}{Cyan},sharp corners,before skip=10pt,after skip=10pt,breakable}
\tcolorboxenvironment{corollary}{boxrule=0pt,boxsep=0pt,colback={violet!10},left=8pt,right=8pt,enhanced jigsaw, borderline west={2pt}{0pt}{violet},sharp corners,before skip=10pt,after skip=10pt,breakable}
\tcolorboxenvironment{proof}{boxrule=0pt,boxsep=0pt,blanker,borderline west={2pt}{0pt}{CadetBlue!80!white},left=8pt,right=8pt,sharp corners,before skip=10pt,after skip=10pt,breakable}
\tcolorboxenvironment{remark}{boxrule=0pt,boxsep=0pt,blanker,borderline west={2pt}{0pt}{Green},left=8pt,right=8pt,before skip=10pt,after skip=10pt,breakable}
\tcolorboxenvironment{remarks}{boxrule=0pt,boxsep=0pt,blanker,borderline west={2pt}{0pt}{Green},left=8pt,right=8pt,before skip=10pt,after skip=10pt,breakable}
\tcolorboxenvironment{example}{boxrule=0pt,boxsep=0pt,blanker,borderline west={2pt}{0pt}{Black},left=8pt,right=8pt,sharp corners,before skip=10pt,after skip=10pt,breakable}
\tcolorboxenvironment{examples}{boxrule=0pt,boxsep=0pt,blanker,borderline west={2pt}{0pt}{Black},left=8pt,right=8pt,sharp corners,before skip=10pt,after skip=10pt,breakable}
\tcolorboxenvironment{cthm}{boxrule=0pt,boxsep=0pt,colback={blue!10},left=8pt,right=8pt,enhanced jigsaw, borderline west={2pt}{0pt}{blue},sharp corners,before skip=10pt,after skip=10pt,breakable}
\tcolorboxenvironment{clemma}{boxrule=0pt,boxsep=0pt,colback={Cyan!10},left=8pt,right=8pt,enhanced jigsaw, borderline west={2pt}{0pt}{Cyan},sharp corners,before skip=10pt,after skip=10pt,breakable}

\usepackage[explicit]{titlesec}
\titleformat{\section}{\fontsize{24}{30}\sffamily\bfseries}{\thesection}{20pt}{#1}
\titleformat{\subsection}{\fontsize{16}{18}\sffamily\bfseries}{\thesubsection}{12pt}{#1}
\titleformat{\subsubsection}{\fontsize{10}{12}\sffamily\large\bfseries}{\thesubsubsection}{8pt}{#1}

\titlespacing*{\section}{0pt}{5pt}{5pt}
\titlespacing*{\subsection}{0pt}{5pt}{5pt}
\titlespacing*{\subsubsection}{0pt}{5pt}{5pt}

\newcommand{\Disp}{\displaystyle}
\newcommand{\qe}{\hfill\(\bigtriangledown\)}
\DeclareMathAlphabet\mathbfcal{OMS}{cmsy}{b}{n}
\setlength{\parindent}{0.2in}
\setlength{\parskip}{0pt}
\setlength{\columnseprule}{0pt}

\definecolor{contcol1}{HTML}{72E094}
\definecolor{contcol2}{HTML}{24E2D6}
\definecolor{convcol1}{HTML}{C0392B}
\definecolor{convcol2}{HTML}{8E44AD}

\usepackage{setspace}
\setstretch{1.25}

\pgfplotsset{soldot/.style={color=black,only marks,mark=*},
  holdot/.style={color=black,fill=white,only marks,mark=*},
compat=1.12}

\counterwithin*{equation}{section}
\counterwithin*{equation}{subsection}

\renewcommand{\epsilon}{\varepsilon}


\begin{document}
\title{MATH 226 Formulae}
\date{Winter Term 1 2023}
\maketitle

\begin{tcolorbox}[title=, fonttitle=\huge\sffamily\bfseries\selectfont,interior style={left color=contcol1!40!white,right color=contcol2!40!white},frame style={left color=contcol1!80!white,right color=contcol2!80!white},coltitle=black,top=2mm,bottom=2mm,left=2mm,right=2mm,drop fuzzy shadow,enhanced,breakable]
  \tableofcontents
\end{tcolorbox}

\newpage
\section{Vectors}
\subsection{Projection}
\begin{definition}
  The vector projection of $\textbf{u}$ onto $\textbf{v}$ is
  \[
    \textbf{u}_\textbf{v} = \frac{\textbf{u} \cdot \textbf{v}}{\norm{\textbf{v}}^2}\textbf{v}
  \] and the scalar projection is \[
    s = \frac{\textbf{u} \cdot \textbf{v}}{\norm{\textbf{v}}}
  \]
\end{definition}
\subsection{Lines and Planes}
\begin{proposition}
  The equation of a plane containing a point $(x_0, y_0, z_0)$ and perpendicular to a vector $(A, B, C)$ can be written as \[
    A(x-x_0) + B(y-y_0) + C(z-z_0) = 0
  \]
\end{proposition}
\begin{proposition}
  The equation of a line containing a point $(x_0, y_0, z_0)$ and with direction vector $(a, b, c)$ can be written as \[
    (x - x_0, y-y_0, z-z_0) = t(a, b, c)
  \]
\end{proposition}
\begin{corollary}
  The above is equivalent to the system of equations \[
    \begin{cases}
      x &= x_0 + at\\
      y &= y_0 + at\\
      z &= z_0 + at\\
    \end{cases}
  \] or
  \[
    (x, y, z) = (a_0, b_0, c_0) + t(a, b, c)
  \]
\end{corollary}

\section{Differentiation}
\subsection{Limits}
\begin{definition}
  If $L \in \R$ and $(a, b)$ is a point in $\R^2$, then $f$ has \textbf{limit} L at $(a, b)$ written as $\lim_{(x, y) \to (a, b)} f(x, y) = L$ if
  \[\forall \ep > 0, \exists \delta > 0, \sqrt{(x-a)^2 + (y-b)^2} < \delta \implies \abs{f(x, y) - L} < \ep\]
\end{definition}
\begin{theorem}
  If the limits along different paths approaching $(a, b)$ are different, then $f$ does not have a limit there.
\end{theorem}
\subsection{Partial Derivatives}
\begin{definition}
  The partial derivative $f$ wrt $x_j$ is the function $f_j$ defined by \[
    f_j(\textbf{x}) = \lim_{h\to0} \frac{1}{h} \left(f(x_1, \dots, x_j + h, \dots, x_n) - f(\textbf{x})\right)
  \]
\end{definition}
\begin{example}
  For 2 variables, the partials are \[
    f_1(x, y) = \lim_{h\to0} \frac{1}{h}(f(x+h,y) - f(x, y))
  \] and \[
    f_2(x, y) = \lim_{h\to0} \frac{1}{h}(f(x,y+h) - f(x, y))
  \]
\end{example}
\subsection{Differentiability}
\begin{definition}
  A function $f(x, y)$ is \textbf{differentiable} if \[
    \lim_{(x, y) \to (a, b)} \frac{\abs{f(x, y) - L(x, y)}}{\sqrt{(x-a)^2+(y-b)^2}} = 0
  \] where \[
    L(x, y) = f(a, b) + f_x(a, b)(x-a) + f_y(a, b)(y-b)
  \]
\end{definition}
\begin{theorem}
  If $f(x, y)$ is $C^1$ on a neighbourhood of $(a, b)$, it is differentiable at $(a, b)$.
\end{theorem}
\begin{theorem}
  If $f$ is differentiable at $(a, b)$, then both $f_x, f_y$ exist at $(a, b)$ and $f$ is continuous at $(a, b)$.
\end{theorem}
\begin{corollary}
  If $f_x, f_y$ exist at $(a, b)$ or $f$ is continuous at $(a, b)$, then $f(x, b)$ is continuous at $x=a$ and $f(a, y)$ is continuous at $y=b$.
\end{corollary}
\subsection{Tangent Planes}
\begin{theorem}
  The normal vector to the tangent plane is \[
    -f_x(a, b)\textbf{i} - f_y(a, b)j + k
  \]
\end{theorem}
\begin{theorem}
  The equation of the tangent plane is \[
    z = f(a, b) + f_x(a, b)(x-a) + f_y(a, b)(y-b)
  \]
\end{theorem}
\begin{proposition}
  The linearization of a function at a point is the same equation as for the tangent plane.
\end{proposition}
\subsection{Directional Derivatives}
\begin{definition}
  The \textbf{directional derivative} of a function in the direction of a unit vector $\textbf{u}$ is defined as \[
    D_\textbf{u}f(a, b) = \lim_{h\to0}\frac{1}{h} (f(a + hu_1, b+hu_2) - f(a, b))
  \]
\end{definition}
\begin{theorem}
  \[D_\textbf{u}f(a, b) = \nabla f(a, b) \cdot \textbf{u} = \abs{\nabla f(a, b)}\cos\theta\]
\end{theorem}
\begin{corollary}
  The direction of steepest ascent is $\textbf{u} = \frac{\nabla f}{\abs{\nabla f}}$, and similarly, the direction of steepest descent is $\textbf{u} = -\frac{\nabla f}{\abs{\nabla f}}$
\end{corollary}
\subsection{Implicit Functions}
\begin{theorem}
  If $F(x, y) = 0$, then if $y = y(x)$, we can use chain rule to yield \[
    0 = \frac{d}{dx} F(x, y(x)) = F_1(x, y(x)) + F_2(x, y(x))\frac{dy}{dx}
  \] so that $\frac{dy}{dx} = -\frac{F_1}{F_2}$. Similarly for $3$ variables, $\frac{\partial z}{\partial x} = -\frac{F_1}{F_3}$ and $\frac{\partial z}{\partial y} = -\frac{F_2}{F_3}$.
\end{theorem}
\begin{theorem}
  If we have the system of equations $u = f(x, y)$ and $v = g(x, y)$, and $x = x(u, v), y=y(u,v)$, we can take the derivative wrt $u$ yielding \[
    1 = f_1(x, y) \cdot \frac{\partial x}{\partial u} + f_2(x, y) \cdot \frac{\partial y}{\partial u}
  \] and \[
    0 = g_1(x, y) \cdot \frac{\partial x}{\partial u} + g_2(x, y) \cdot \frac{\partial y}{\partial u}
  \]
  We can then solve for $\frac{\partial x}{\partial u}$ and $\frac{\partial y}{\partial u}$.
\end{theorem}
\subsection{Minimum and Maximum}
\begin{cthm}[Conditions for a local min/max]
  We must have at least one of
  \begin{enumerate}
    \item $\nabla f(a, b)= 0$
    \item $\nabla f(a, b)$ DNE
    \item $(a, b)$ is a boundary point of the domain $D$
  \end{enumerate}
\end{cthm}
\begin{definition}[Taylor Polynominals]
  We have \[
    f(\textbf{x}) \approx f(\textbf{a}) + \nabla f(\textbf{a}) \cdot (\textbf{x}-\textbf{a}) + \frac{1}{2} (\textbf{x}-\textbf{a}) \cdot \mathscr{H}_{f(\textbf{a})}(\textbf{x}-\textbf{a})^t
  \] where \[
    \mathscr{H} =
    \begin{pmatrix}
      f_{11} & f_{12} & \dots & f_{1n}\\
      \vdots & \vdots & \vdots & \vdots\\
      f_{n1} & f_{n2} & \dots & f_{nn}\\
    \end{pmatrix}
  \]
\end{definition}
\begin{example}
  For a polynominal in 2 variables, \[
    p_2(x, y) = f(a, b) + f_1(a, b)(x-a) + f_2(a, b)(y-b) + \frac{1}{2}f_{11}(a, b)^2 + f_{12}(a, b)(x-a)(y-b) + \frac{1}{2}f_{22}(a, b)(y-b)^2
  \]

\end{example}
\begin{cthm}[Classifying Critical Points with the Second Derivative Test]
  We consider the behaviour of the \textbf{principal minors} of $\mathscr{H}$ evaluated at $\textbf{a}$.
  We have \[
    D_1 = f_{11}, D_{2} =
    \begin{vmatrix}
      f_{11} & f_{12}\\
      f_{21} & f_{22}
    \end{vmatrix}, D_3 =
    \begin{vmatrix}
      f_{11} & f_{12} & f_{13}\\
      f_{21} & f_{22} & f_{23}\\
      f_{31} & f_{32} & f_{33}
    \end{vmatrix}
  \]

  \begin{enumerate}
    \item If all $D_i > 0$, then $\mathscr{H}$ is positive definite and so $f$ has a \textbf{local minimum} at $\textbf{a}$
    \item If $D_i < 0$ when $i$ is even and $D_i > 0$ when $i$ odd, then $\mathscr{H}$ is negative definite and so $f$ has a \textbf{local maximum} at $\textbf{a}$
    \item If $D_n = \det(\mathscr{H}) \neq 0$ (the largest one) but the sign is not all positive or alternating with negative first, then $\mathscr{H}$ is indefinite and so $f$ has a \textbf{saddle point} at \textbf{a}
    \item If $D_n = 0$, then the test is inconclusive
  \end{enumerate}
\end{cthm}
\begin{definition}
  A domain $X \subset \R^n$ is \textbf{compact} if it is bounded and closed.
\end{definition}
\begin{cthm}[Procedure for finding Global Extrema on Compact Domains]
  If $X$ is compact and $f$ cts., then it must attain a global min/max on $X$.
  \\
  If a point is a global minimum, it has to be a local one as well or on the boundary, thus we can follow a similar process as before.

  \begin{enumerate}
    \item Find all critical/singular points inside $X$ of the function $f$
    \item Determine the min/max values of $f$ on the boundary of $X$ and note the points where these min/max values are attained
    \item Compute $f(\textbf{a})$ at all the points found, and pick the biggest and smallest
  \end{enumerate}

  Easier to just compute and not use 2nd derivative test.
\end{cthm}
\subsection{Lagrange Multipliers}
\begin{remark}
  We want to determine the extreme values of $f$ subject to a constraint $g(\textbf{x}) = c$. We want $\nabla f$ parallel to $\nabla g$.
\end{remark}
\begin{theorem}
  Want to find all points where
  \begin{itemize}
    \item $\nabla f(x, y) = \lambda \nabla g(x, y)$ for some $\lambda \in \R$
    \item $\nabla g(x, y) = 0$
    \item $\nabla f(x, y)$ or $\nabla g(x, y)$ DNE
  \end{itemize}
\end{theorem}
\section{Integration}
\begin{cthm}[Determining Convergence of Improper Integrals]
  Let $D \subset \R^2$ be a region.
  \begin{itemize}
    \item If $f \geq g \geq 0$ on D, then $
      \iint_D g(x, y) dA = \infty \implies \iint_D f(x, y) = \infty$.
    \item If $0 \leq f \leq g$ on D, then $
      \iint_D g(x, y) < \infty \implies \iint_D f(x, y) < \infty$.
  \end{itemize}
\end{cthm}
\begin{proposition}
  A double integral over a region $D$ can be interpreted as \[
    \iint_D f(x, y) dA = \text{volume under the graph of } z = f(x, y) \text{ above }D
  \]
\end{proposition}
\begin{theorem}
  The average value of $f(x, y)$ on $F$ is \[
    \frac{\iint_D f(x,y) dA}{\iint_D 1 dA}
  \] since $\iint_D 1 dA$ is the area of $D$.
\end{theorem}
\begin{definition}
  The \textbf{centroid} of $D$ is the point $(\overline{x}, \overline{y})$ where \[
    \overline{x} = \frac{\iint_D x dA}{\iint_D 1 dA}
  \] and similarly for $\overline{y}$.
\end{definition}
\subsection{Cylindrical Coordinates}
\begin{definition}
  Cylindrical coordinates are obtained by the change of variables $(x, y) \to (r, \theta)$, where \[
    x = r\cos\theta, y=r\sin\theta
  \] and \[
    dV = dxdydz = rdrd\theta dz
  \]
\end{definition}
\subsection{Spherical Coordinates}
\begin{definition}
  Consider a point $P$. Then $P$ can be represented uniquely by
  \begin{itemize}
    \item $R$ is the radius of the sphere
    \item $0 \leq \theta \leq 2\pi$ is the projection of the point onto the $xy$-axis and taking the angle from the $x$ axis to $P'$.
    \item $0 \leq \phi \leq \pi$ is the angle made by $P$ and the $z$-axis.
  \end{itemize}
\end{definition}
\subsection{Change of Variables}
\begin{theorem}
  The conversion for area is \[
    dA = \abs{\frac{\partial(x, y)}{\partial(u, v)}} =
    \begin{vmatrix}
      x_u & x_v\\
      y_u & v_v\\
    \end{vmatrix}
  \]

  It is also equivalent to \[
    \left(\frac{\partial(u, v)}{\partial(x, y)}\right)^{-1}
  \]
\end{theorem}
\begin{remark}
  Remember to flip and take the absolute value!!!
\end{remark}
\end{document}
